\chapter*{Introduzione}

Questo elaborato ha come scopo quello di dimostrare l'esistenza della soluzione generale del problema di Riemann, ossia trovare una soluzione debole del seguente sistema di leggi di conservazione
\begin{equation*}
    \partial_{t}u+\partial_{x}f(u)=0,
\end{equation*}
con dato iniziale una funzione costante a tratti, della forma
\begin{equation*}
    u(0,x)=
    \begin{cases}
        u^{-}\text{ se }x<0,\\
        u^{+}\text{ se }x>0.
    \end{cases}
\end{equation*}

Esso è suddiviso in due capitoli.\\
Nel primo, preliminare per il secondo, lo scopo sarà quello di enunciare e dimostrare il teorema della funzione implicita nel caso generale in cui si ha dipendenza da un parametro. Per fare ciò avremo però bisogno di altri teoremi preliminari, tra i quali il teorema delle contrazioni e il teorema della funzione inversa (sia senza dipendenza da una parametro, sia con).\\
Difatti questi teoremi verranno utilizzati nel capitolo successivo e saranno necessari per alcune dimostrazioni, compresa quella dell'esistenza della soluzione generale del problema di Riemann.\\
Nel secondo capitolo affronteremo il problema di Riemann. Inizieremo facendo una breve introduzione ai sistemi di leggi di conservazione, dando qualche definizione più tecnica. In seguito tratteremo di quella che viene chiamata catastrofe del gradiente che, in un modo che vedremo, rende necessaria l'elaborazione di una nuova nozione di soluzione per una equazione alle derivate parziali; entrano infatti in gioco le cosiddette soluzioni deboli, ossia soluzioni che non soddisfano puntualmente l'equazione ma la soddisfano in una maniera meno ``restrittiva", nozione fondamentale per trattare le PDE.\\ Introdurremo quindi il problema di Riemann, dando inoltre la definizione di sistema strettamente iperbolico e dimostrandone alcune proprietà. A questo punto ricaveremo e descriveremo delle soluzioni deboli particolari del problema di Riemann: le onde di rarefazione e le onde di shock (curve particolari che soddisfano le condizioni di Rankine-Hugoniot).\\
Infine, utilizzando le due particolari soluzioni deboli viste in precedenza, costruiremo e dimostreremo l'esistenza della soluzione generale del problema di Riemann.\\
Seppure così descritto sembra un problema molto astratto e artificioso, questo tipo di PDE compare molto spesso in ambiti della fisica e della matematica. Nell'elaborato, ad esempio, mostreremo due sue applicazioni allo studio del flusso del traffico stradale e alla fluidodinamica.\\
Gran parte delle definizioni, degli enunciati dei teoremi e delle dimostrazioni sono tratte da \cite{bressan_2000}.